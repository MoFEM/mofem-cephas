\documentclass{article}
\usepackage[utf8]{inputenc}
\usepackage{amsmath,amsfonts,amssymb}
\usepackage{graphicx}
\usepackage{times}
\parindent0pt
\parskip 1.5ex plus 1ex minus .5ex
%%%%%%%%%%%%%%%%%%%%%%%%%%%%%%%%%%%%%%%%%%%%%%%%%%%%%%%%%%%%%%%%%%%%%%
% Do not change above this line! Do not add packages! You may comment
% out packages missing in your installation.
%
% Everything before "\begin{document}" will be ignored! Do not define
%   macros!
%%%%%%%%%%%%%%%%%%%%%%%%%%%%%%%%%%%%%%%%%%%%%%%%%%%%%%%%%%%%%%%%%%%%%%
\begin{document}
\title{Implementation in MoFEM of hierarchical and heterogenous approximations bases}
\author{%
  Lukasz Kaczmarczyk and Chris Pearce,\\
  \em{University of Glasgow},\\
  \texttt{lukasz.kaczmarczyk@glasgow.ac.uk}\\
  \texttt{chris.pearce@glasgow.ac.uk}\\
}
\maketitle

MoFEM~\cite{MoFEMWebPage} (Mesh Oriented Finite Element Method) is a C++ library
supporting the solution of finite elements problems. It is developed to provide
a free and open  finite element code for engineers, students and academics. It
is tailored for the solution of multi-physics problems with arbitrary levels of
approximation, different levels of mesh refinement and optimised for
high-performance computing. It is also designed to be able manage complexities
related to heterogeneous order of approximations for L2,H1,H-div and H-curl
spaces, see~\cite{ainsworth2001essential}.

MoFEM is the blend of the Boost MultiIndex containers, MOAB (Mesh Oriented
Database)~\cite{tautges2010canonical} and PETSc (Portable, Extensible Toolkit
for Scientific Computation)~\cite{petsc-web-page}. It is open-source software
under the GNU Lesser General Public License. MoFEM can read and write a number
of different mesh file formats using functionality provided by MoAB. The current
version of MoFEM has full support for CUBIT/TRELIS, TetGEN and NetGEN for
pre-processing and ParaView for post-processing.

The MoFEM library itself is designed to be small, modular and extendable. Users
can implement modules as independent projects, located in its own repository,
which can be either public or private, with their own copyright and license.
Such flexibility, allowing for privacy and different licensing parts of the
code, is designed to accommodate the needs of both academics and industry. Since
each user module is implemented in MoFEM using its internal database, data
between modules can be exchanged without prior knowledge of how each module is
developed.

We will present various implementations of problems involving hierarchical
approximations spaces focussing attention on implementation and general
philosophy of the code.

% begin{thebibliography}{10}
%
% \bibitem{PDESoft}
% P.~Bastian, V.~Heuveline, G.~Kanschat, A.~Reuter, PDESoft 2014
% Proceedings, \emph{Archive of Numerical Software} \textbf{2}
% (2014), 1234--4321.
% \end{thebibliography}

\bibliography{references}{}
\bibliographystyle{plain}

\end{document}

%%% Local Variables:
%%% mode: latex
%%% TeX-master: t
%%% End:
