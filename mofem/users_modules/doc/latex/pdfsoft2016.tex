\documentclass{article}
\usepackage[utf8]{inputenc}
\usepackage{amsmath,amsfonts,amssymb}
\usepackage{graphicx}
\usepackage{times}
\parindent0pt
\parskip 1.5ex plus 1ex minus .5ex
%%%%%%%%%%%%%%%%%%%%%%%%%%%%%%%%%%%%%%%%%%%%%%%%%%%%%%%%%%%%%%%%%%%%%%
% Do not change above this line! Do not add packages! You may comment
% out packages missing in your installation.
%
% Everything before "\begin{document}" will be ignored! Do not define
%   macros!
%%%%%%%%%%%%%%%%%%%%%%%%%%%%%%%%%%%%%%%%%%%%%%%%%%%%%%%%%%%%%%%%%%%%%%
\begin{document}
\title{Hierarchical and heterogenous approximations bases on heterogenous messes - implementation in MoFEM}
\author{%
  Lukasz Kaczmarczyk and Chris Pearce,\\
  \em{University of Glasgow},\\
  \texttt{lukasz.kaczmarczyk@glasgow.ac.uk}\\
  \texttt{chris.pearce@glasgow.ac.uk}\\
}
\maketitle

MoFEM~\cite{MoFEMWebPage} (Mesh Oriented Finite Element Method) it is C++ library supporting
solution of finite elements problems. Is developed to o provide free and open,
inclusive, participation community finite element code for engineers, students
and academics.

MoFEM is a finite element analysis code tailored for the solution of
multi-physics problems with arbitrary levels of approximation, different levels
of mesh refinement and optimised for high-performance computing. It is design to
be able manage complexities related to heterogeneous order of approximations for
L2,H1,H-div and H-curl spaces, see~\cite{ainsworth2001essential}.

MoFEM is the blend of the Boost MultiIndex containers, MOAB (Mesh Oriented
Database)~\cite{tautges2010canonical} and PETSc (Portable, Extensible Toolkit
for Scientific Computation)~\cite{petsc-web-page}. MoFEM is developed in C++ and
it is open-source software under the GNU Lesser General Public License.

MoFEM can read and write number of mesh file formats using functionality
provided by MoAB. The current version of MoFEM has full support for
CUBIT/TRELIS, TetGEN, NetGEN for pre-processing and ParaView for
post-processing.

MoFEM library itself is designed to be small, modular and extendable. Itself it
is not library of finite elements solving particular problems of continuum
mechanics, f.e. problem of thermal conduction. That is implemented in users
modules using MoFEM. User module in principle is independent project, located in
its own repository, which can be public or private. Can have is own copyright
and license. Such flexibility, allowing for privacy and different licensing
parts of the code, is designed for needs both of academics add industry. Since
each user module is implemented with MoFEM, using its internal database, data
between modules can be exchanged without prior knowledge how each module is
developed.

% begin{thebibliography}{10}
%
% \bibitem{PDESoft}
% P.~Bastian, V.~Heuveline, G.~Kanschat, A.~Reuter, PDESoft 2014
% Proceedings, \emph{Archive of Numerical Software} \textbf{2}
% (2014), 1234--4321.
% \end{thebibliography}

\bibliography{references}{}
\bibliographystyle{plain}

\end{document}

%%% Local Variables:
%%% mode: latex
%%% TeX-master: t
%%% End:
